\mode*

\chapter{Some examples}

\section{Grade statistics}

We want to compute some statistics for the grade distributions.
We can do this as follows.
\begin{pyblock}[statstest][numbers=left]
import numpy as np
import pandas as pd

prgi = pd.read_csv("DD1315.csv")

rounds_years = {
  "51386": "2020",
  "50869": "2019",
  "50662": "2018",
  "50650": "2017",
  "50523": "2016",
  "50917": "2015",
  "51210": "2014",
  "50094": "2013",
  "50167": "2012",
  "60420": "2011"
}
prgi_LAB3 = prgi[
  prgi.Component.eq("LAB3") & prgi.Round.isin(rounds_years.keys())
].replace(rounds_years)

stats = pd.crosstab(prgi_LAB3.Grade, prgi_LAB3.Round)

for column in stats:
  stats[column] /= stats[column].sum()

print(stats.round(3).to_latex(
  label="GradeDistribution",
  caption="Grade distribution for the DD1315 course."
))
\end{pyblock}
This yields \cref{GradeDistribution}.

\stdoutpythontex

We can also show this as a bar plot.
\begin{pyblock}[statstest][numbers=left,firstnumber=last]
import matplotlib
import matplotlib.pyplot as plt
import tikzplotlib

stats.plot.bar()

print(tikzplotlib.get_tikz_code(axis_width="13cm"))
\end{pyblock}
This yields \cref{GradeDistributionFig}.

\begin{figure}
  \centering
  \stdoutpythontex
  \caption{Grade distribution for the DD1315 course.}
  \label{GradeDistributionFig}
\end{figure}

If we convert the grades to numbers, then we can also compute the average grade 
with standard deviation \etc.
\begin{pyblock}[statstest][numbers=left,firstnumber=last]
grade_map = {
  "A": 5, "B": 4, "C": 3, "D": 2, "E": 1, "-": 0, np.nan: 0
}
grades_num = prgi_LAB3[ ["Round", "Grade"] ].replace(grade_map)

print(grades_num.groupby("Round")["Grade"].describe().round(3).to_latex(
  label="GradeStats",
  caption="Statistics for grades of the DD1315 course, grade map: " +
    ", ".join([f"{g} $\mapsto$ {s}" for g, s in grade_map.items()]) + "."
))
\end{pyblock}
This yields \cref{GradeStats}.

\stdoutpythontex

We'll try to visualize the density of the grade distribution.
\begin{pyblock}[statstest][numbers=left,firstnumber=last]
grades_num.groupby("Round").plot.density()

print(tikzplotlib.get_tikz_code(axis_width="13cm"))
\end{pyblock}
This yields \cref{GradeDensityFig}.

\begin{figure}
  \centering
  \stdoutpythontex
  \caption{Grade density for the DD1315 course.}
  \label{GradeDensityFig}
\end{figure}

Finally, we'd like to merge the old rounds to compare with the latest.
We can see in \cref{GradeDistribution,GradeStats} that the years 2019--2016 and 
2015--2011 seem to be clustered in terms of performance.
They have similar numbers in those clusters.
\begin{pyblock}[statstest][numbers=left,firstnumber=last]
rounds_avg = {
  "2019": "2019--2016",
  "2018": "2019--2016",
  "2017": "2019--2016",
  "2016": "2019--2016",
  "2015": "2015--2011",
  "2014": "2015--2011",
  "2013": "2015--2011",
  "2012": "2015--2011",
  "2011": "2015--2011"
}
grades_avg = grades_num.replace(rounds_avg)

print(grades_avg.groupby("Round")["Grade"].describe().round(3).to_latex(
  label="GradeStatsAvg",
  caption="Statistics for grades of the DD1315 course, grade map: " +
    ", ".join([f"{g} $\mapsto$ {s}" for g, s in grade_map.items()]) + "."
))
\end{pyblock}
This yields \cref{GradeStatsAvg}.

\stdoutpythontex

We can also plot this.
\begin{pyblock}[statstest][numbers=left,firstnumber=last]
grades_grouped = prgi_LAB3[ ["Round", "Grade"] ].replace(rounds_avg)

stats_grouped = pd.crosstab(grades_grouped.Grade, grades_grouped.Round)

for column in stats_grouped:
  stats_grouped[column] /= stats_grouped[column].sum()

stats_grouped.plot.bar()

print(tikzplotlib.get_tikz_code(axis_width="13cm"))
\end{pyblock}
This yields \cref{GradeDistGroupedFig}.

\begin{figure}
  \centering
  \stdoutpythontex
  \caption{Grade distribution grouped over the years.}
  \label{GradeDistGroupedFig}
\end{figure}

\endinput

\section{Example time analyses}

\begin{frame}
  \begin{figure}
    \input{DD1315-LAB1-time.pgf}
    \caption{Distribution of finishing times for LAB1 on DD1315.}
    \label{DD1315-LAB1-time-fig}
  \end{figure}
\end{frame}

\begin{frame}
  \begin{figure}
    \input{DD1315-LAB2-time.pgf}
    \caption{Distribution of finishing times for LAB2 on DD1315.}
    \label{DD1315-LAB2-time-fig}
  \end{figure}
\end{frame}

\begin{frame}
  \begin{figure}
    \input{DD1315-LAB3-time.pgf}
    \caption{Distribution of finishing times for LAB3 on DD1315.}
    \label{DD1315-LAB3-time-fig}
  \end{figure}
\end{frame}


\section{Example grade analyses}

We performed some analyses on the courses DD1315, DD1310, DD1312.
DD1315 is covered in \cref{%
  DD1315-all-figure,DD1315-all-table,%
  DD1315-grouped-figure,DD1315-grouped-table,%
  DD1315-last-figure,DD1315-last-table%
}.
DD1310 is covered in \cref{%
  DD1310-all-figure,DD1310-all-table,%
  DD1310-grouped-figure,DD1310-grouped-table,%
}.
DD1312 is covered in \cref{%
  DD1312-all-figure,DD1312-all-table,%
  DD1312-grouped-figure,DD1312-grouped-table,%
}

\begin{frame}
\begin{figure}
\centering
\input{DD1315-all.pgf}
\caption{All rounds of DD1315 separately.}
\label{DD1315-all-figure}
\end{figure}
\end{frame}

\begin{table}
\centering
\input{DD1315-all.tex}
\caption{All rounds of DD1315 separately.}
\label{DD1315-all-table}
\end{table}

\begin{figure}
\centering
\input{DD1315-grouped.pgf}
\caption{Rounds of DD1315 suitably grouped.}
\label{DD1315-grouped-figure}
\end{figure}

\begin{table}
\centering
\input{DD1315-grouped.tex}
\caption{Rounds of DD1315 suitably grouped.}
\label{DD1315-grouped-table}
\end{table}

\begin{figure}
\centering
\input{DD1315-last.pgf}
\caption{All rounds of DD1315 grouped together, except the last.}
\label{DD1315-last-figure}
\end{figure}

\begin{table}
\centering
\input{DD1315-last.tex}
\caption{All rounds of DD1315 grouped together, except the last.}
\label{DD1315-last-table}
\end{table}

\begin{figure}
\centering
\input{DD1310-all.pgf}
\caption{All rounds of DD1310 separately.}
\label{DD1310-all-figure}
\end{figure}

\begin{table}
\centering
\input{DD1310-all.tex}
\caption{All rounds of DD1310 separately.}
\label{DD1310-all-table}
\end{table}

\begin{figure}
\centering
\input{DD1310-grouped.pgf}
\caption{All rounds of DD1310 grouped suitably together.}
\label{DD1310-grouped-figure}
\end{figure}

\begin{table}
\centering
\input{DD1310-grouped.tex}
\caption{All rounds of DD1310 grouped suitably together.}
\label{DD1310-grouped-table}
\end{table}

\begin{figure}
\centering
\input{DD1312-all.pgf}
\caption{All rounds of DD1312 separately.}
\label{DD1312-all-figure}
\end{figure}

\begin{table}
\centering
\input{DD1312-all.tex}
\caption{All rounds of DD1312 separately.}
\label{DD1312-all-table}
\end{table}

\begin{figure}
\centering
\input{DD1312-grouped.pgf}
\caption{All rounds of DD1312 grouped suitably together.}
\label{DD1312-grouped-figure}
\end{figure}

\begin{table}
\centering
\input{DD1312-grouped.tex}
\caption{All rounds of DD1312 grouped suitably together.}
\label{DD1312-grouped-table}
\end{table}


\chapter{The main CLI program}

\input{../src/latools/cli/cli.tex}


\appendix

\input{grades.tex}
\input{time.tex}
